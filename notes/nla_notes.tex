\documentclass[12pt,a4paper]{book}

\usepackage[left=2cm,right=2cm,top=2.5cm,bottom=2.5cm,ignorefoot=true]{geometry}

\setlength{\parindent}{2em}
\setlength{\parskip}{0.5em}
\renewcommand{\baselinestretch}{1.25}

\usepackage{fancyhdr}
\pagestyle{fancy}
\fancyhead{}
\renewcommand{\sectionmark}[1]{\markright{\thesection. #1}}
\renewcommand\chaptermark[1]{\markboth{#1}{}\def\cchapter{#1}}
\fancyhead[RE]{\leftmark}
\fancyhead[LO]{\rightmark}
\fancyhead[RO,LE]{\thepage}
\fancyfoot{}
\renewcommand{\footrulewidth}{0.2pt}
\fancyfoot[C]{{\em \small NLAAI - Notes}}

\makeatletter 
\def\clearpage{% 
	\ifvmode 
		\ifnum \@dbltopnum =\m@ne 
			\ifdim \pagetotal < \topskip 
				\hbox{} 
			\fi 
		\fi 
	\fi 
\newpage 
\thispagestyle{empty} 
\write\m@ne{} 
\vbox{} 
\penalty -\@Mi 
} 
\makeatother

\usepackage[dvips]{graphicx}
\usepackage{titlesec,color}
\usepackage[font=small,labelfont=bf]{caption}

\definecolor{gray75}{gray}{0.5}
\newcommand{\hsp}{\hspace{20pt}}
\titleformat{\chapter}[hang]{\huge\bfseries}{\thechapter\hsp\textcolor{gray75}{$|$}\hsp}{0pt}{\huge\bfseries}

\usepackage[utf8]{inputenc}
\usepackage[english]{babel}
\usepackage{amssymb,amsfonts,latexsym,amsmath,amscd}
\usepackage{lmodern}

%---------------
% Environments
\usepackage{amsthm}
\usepackage[most,theorems]{tcolorbox}
\definecolor{block-gray}{gray}{0.75}
\newtcolorbox{blockquote}{colback=block-gray,grow to right by=-1mm,grow to left by=-1mm,boxrule=0pt,boxsep=0pt,breakable}

\newtcbtheorem[number within=section]%
{definition} % \begin..
{Definition} % Title
{fonttitle=\bfseries,colframe=red!75!black,colback=red!5!white,breakable} % Style 
{def} % label prefix; cite as ``def:yourlabel''

\newtcbtheorem[number within=section]%
{theorem} % \begin..
{Theorem} % Title
{fonttitle=\bfseries,colframe=blue!75!black,colback=blue!5!white,breakable} % Style 
{teo} % label prefix; cite as ``teo:yourlabel''

\newtcbtheorem[number within=section]%
{proposition} % \begin..
{Proposition} % Title
{fonttitle=\bfseries,colframe=blue!75!black,colback=blue!5!white,breakable} % Style 
{prop} % label prefix; cite as ``prop:yourlabel''

\newtcbtheorem[number within=section]%
{corollary} % \begin..
{Corollary} % Title
{fonttitle=\bfseries,colframe=blue!75!black,colback=blue!5!white,breakable} % Style 
{cor} % label prefix; cite as ``cor:yourlabel''

\newtcbtheorem[number within=section]%
{notation} % \begin..
{Notation} % Title
{fonttitle=\bfseries,colframe=green!75!black,colback=green!5!white,breakable} % Style 
{note} % label prefix; cite as ``nota:yourlabel''

\newtcbtheorem[number within=section]%
{lemma} % \begin..
{Lemma} % Title
{fonttitle=\bfseries,colframe=yellow!75!black,colback=yellow!5!white,breakable} % Style 
{lemma} % label prefix; cite as ``nota:yourlabel''

%-----------------------
\newcommand {\tr}{\mathrm{tr}}
\newcommand {\R}{\mathbb{R}}
\newcommand{\PP}{\mathbb{P}}
\newcommand{\E}{\mathbb{E}}
\renewcommand {\H}{\mathbb{H}}
\newcommand {\1}{\textrm{\textbf{1}}}
\newcommand {\Id}{\mathscr{I}}
\newcommand {\F}{\mathcal{F}}
\newcommand {\G}{\mathcal{G}}
\newcommand {\I}{\mathcal{I}}
\newcommand {\N}{\mathcal{N}}
\newcommand {\C}{\mathcal{C}}
\newcommand {\T}{\mathcal{T}}
\renewcommand {\P}{\mathcal{P}}
\renewcommand {\L}{\mathcal{L}}
\newcommand {\FF}{\mathscr{F}}
\newcommand {\BB}{\mathscr{B}}
\newcommand {\KK}{\mathscr{K}}
\newcommand {\RR}{\mathscr{R}}
\newcommand {\GG}{\mathscr{G}}
\newcommand {\TT}{\mathscr{T}}
\newcommand {\WW}{\mathscr{W}}
\newcommand {\Pj}{\mathscr{P}}
\newcommand {\CC}{\mathscr{C}}
\newcommand {\A}{\mathscr{A}}
\newcommand {\Lo}{\mathscr{L}}

\newcommand{\der}[1]{#1^{\prime}}
\newcommand{\abs}[1]{\lvert#1\rvert}
\newcommand{\norm}[1]{\lVert#1\rVert}
\newcommand{\inp}[1]{\langle#1\rangle}
\newcommand{\conj}[1]{\lbrace#1\rbrace}
\newcommand{\roof}[1]{\lceil#1\rceil}
\newcommand{\mx}[1]{max\lbrace#1\rbrace}
\newcommand{\mn}[1]{min\lbrace#1\rbrace}
\newcommand{\Lah}{L_{a}^{1}(h)}
\newcommand{\mlp}[1]{\textsl{MLP}(\rho,d,#1)}

\title{\textsc{Numerical Linear Algebra: An Introduction - Notes}}
\author{Enki A. Barra Melendrez}
\date{}

\begin{document}
\maketitle
\setcounter{section}{1}
\section{Error, Stability and Conditioning}

\begin{definition}{Floating point number}
    A B-adic, normalized floating point number of precision $m$ is either $x=0$ or:
    \begin{equation*}
        x = B^e \sum_{k=-m}^{-1} x_k B^{-k}, \ x_{-1} \neq 0, \ x_k \in \{0,1,\ldots,B-1\}
    \end{equation*}
    Where: 
    \begin{itemize}
        \item $B \geq 2$ is the base of the number system.
        \item $e_{min} \leq e \leq e_{max}$ is the exponent.
        \item $\sum_{k=-m}^{-1} x_k B^{-k}$ is the mantissa.
    \end{itemize}
\end{definition}

\noindent Many programming languages use the IEEE 754 standard for floating point arithmetic. In this standard, for a double precision number, the base is $B=2$, the mantissa has $m=52$ bits and the exponent has $11$ bits.\\
\\
\noindent Two additional numbers are added to the set of floating point numbers: $\pm \infty$ and NaN (Not a Number) which is used to represent undefined or unrepresentable values.\\

\begin{definition}{Machine epsilon}
    The machine epsilon $eps$ is the smallest positive number which satisfies:
    \[
       | x -rd(x) | \leq eps | x |
    \]
    Where $rd(x)$ is the floating point representation of $x$. 
\end{definition}
\noindent Usually this rounding function is taken to be the nearest machine number. A B-system with precision $m$ the associated the machine epsilon is given by:\\
\begin{theorem}{Floating point machine epsilon}
    For a floating point number system with base $B$ and precision $m$, the machine epsilon is given by:
    The machine epsilon is given by $eps = B^{1-m}$ , i.e we have: 
    \[
        | x -rd(x) | \leq B^{1-m} | x |
    \]
\end{theorem}

\begin{theorem}{Floating point ope}
    Let $\star$ be one of the operations $+, -, \times, /$ and let $\circledast$ be the equivalent floating point operation, then $\forall x, y$ in the floating point system, there exists an $\epsilon$ such that:
    \[
        x \star y = (x \circledast y)(1 + \epsilon).
    \]
\end{theorem}

\begin{definition}{Compatibility of matrix norms}
  Given the norms $\| \cdot \|_{(n)}$ and $\| \cdot \|_{(m)}$ on $\R^n$ and $\R^{m}$ respectively, we say that a matrix norm $\| \cdot \|_{\star}$ is \textbf{compatible} with these norms if:
  \[
      \| A x \|_{(m)} \leq \| A \|_{\star} \| x \|_{(n)}, \quad \forall x \in \R^n.
  \]
\end{definition}

\noindent \textbf{Matrix-vector multiplication} arises naturally when solving linear systems of equations. Given a matrix $\mathbf{A} \in \R^{m \times n}$, $x, b \in \R^n$. Let us consider the situation
\begin{align*}
    b &= \mathbf{A} x \text{,} \\
    \delta b &= \mathbf{A} \delta x
\end{align*}

\noindent If the matix $\mathbf{A}: \R^n \to \R^n$ is non-singular, then $\| x \|_{\star} := \| \mathbf{A} x \|$ is a norm on $\R^n$. From the fact that all norms 
in $\R^n$ are equivalent follows that there exits two constants $C_1$ and $C_2$ such that:\\
\[
    C_1 \| x \| \leq \|\mathbf{A} x\| \leq C_2 \| x \| \implies \frac{\| \delta b \|}{\| b \|} \leq \frac{C_2}{C_1} \frac{\| \delta x \|}{\| x \|}.
\]
\bibliography{bib}
\end{document}
