\documentclass[a4paper, 12pt]{article}

\usepackage{geometry}
\geometry{dvips,a4paper,margin=1.2in}
\usepackage{fancyhdr}
\usepackage{lipsum}
\pagestyle{fancy}
\headheight=17pt
\fancyhf{}
\chead{Notes on NLA}
\renewcommand{\headrulewidth}{0pt}
\cfoot{\thepage}

\usepackage{amsmath,amsfonts,amscd,amssymb,amsthm, mathrsfs}
\usepackage{dsfont}
\usepackage{longtable}
\usepackage[english]{babel}
\usepackage[utf8]{inputenc}
\usepackage[active]{srcltx}
\usepackage[T1]{fontenc}
\usepackage{graphicx}
\usepackage{enumitem}
\usepackage{bbm}
\usepackage{enumerate}   
\usepackage{mathtools}
\usepackage[colorlinks = true,
            linkcolor = blue,
            urlcolor  = magenta,
            citecolor = blue,
            anchorcolor = blue]{hyperref}
\usepackage{subfig}
\usepackage{MnSymbol}
\usepackage{stmaryrd}
\usepackage{nicefrac}
\usepackage{natbib}
\usepackage{rotating}
\usepackage{xcolor}
\usepackage{framed}
\usepackage[affil-it]{authblk}
\usepackage{verbatim}

\colorlet{shadecolor}{blue!14}

%%%%%%%%%%%%%%%%%%%%%%%%%%%%%%%%%%%%%%%%%%%%%%% environments %%%%%%%%%%%%%%%%%%%%%%%%%%%%%%

\newtheorem{theorem}{Theorem}[section]
\newtheorem{proposition}{Proposition}[section]
\newtheorem{corollary}{Corollary}[section]
\newtheorem{lemma}{Lemma}[section]
\newtheorem{definition}{Definition}[section]


%%%%%%%%%%%%%%%%%%%%%%%%%%%%%%%%%%%%%%%%%%%%%%% commands %%%%%%%%%%%%%%%%%%%%%%%%%%%%%%
\newcommand {\tr}{\mathrm{tr}}
\newcommand {\R}{\mathbb{R}}
\newcommand{\PP}{\mathbb{P}}
\newcommand{\E}{\mathbb{E}}
\renewcommand {\H}{\mathbb{H}}
\newcommand {\1}{\textrm{\textbf{1}}}
\newcommand {\Id}{\mathscr{I}}
\newcommand {\F}{\mathcal{F}}
\newcommand {\G}{\mathcal{G}}
\newcommand {\I}{\mathcal{I}}
\newcommand {\N}{\mathcal{N}}
\newcommand {\C}{\mathcal{C}}
\newcommand {\T}{\mathcal{T}}
\renewcommand {\P}{\mathcal{P}}
\renewcommand {\L}{\mathcal{L}}
\newcommand {\FF}{\mathscr{F}}
\newcommand {\BB}{\mathscr{B}}
\newcommand {\KK}{\mathscr{K}}
\newcommand {\RR}{\mathscr{R}}
\newcommand {\GG}{\mathscr{G}}
\newcommand {\TT}{\mathscr{T}}
\newcommand {\WW}{\mathscr{W}}
\newcommand {\Pj}{\mathscr{P}}
\newcommand {\CC}{\mathscr{C}}
\newcommand {\A}{\mathscr{A}}
\newcommand {\Lo}{\mathscr{L}}
\newcommand{\range}{\mathrm{range}}
\newcommand{\Frechet}{Fréchet }
\newcommand{\Tan}{\T}

\newcommand{\red}[1]{{\color{red} #1}}
\newcommand{\blu}[1]{{\color{blue} #1}} 

\usepackage{enumitem, hyperref}
\makeatletter
\def\namedlabel#1#2{\begingroup
  #2%
  \def\@currentlabel{#2}%
  \phantomsection\label{#1}\endgroup
}
\makeatother

\title{\Large{Numerical Linear Algebra: An Introduction - Notes}}
\author{
  Enki A. Barra Melendrez
}
\date{\today}

\begin{document}

\maketitle
\setcounter{section}{1}

\section{Error, Stability and Conditioning}

\begin{definition}
    A B-adic, normalized floating point number of precision $m$ is either $x=0$ or:\\
    \begin{equation*}
        x = B^e \sum_{k=-m}^{-1} x_k B^{-k}, \ x_{-1} \neq 0, \ x_k \in \{0,1,\ldots,B-1\}
    \end{equation*}
    Where: 
    \begin{itemize}
        \item $B \geq 2$ is the base of the number system.
        \item $e_{min} \leq e \leq e_{max}$ is the exponent.
        \item $\sum_{k=-m}^{-1} x_k B^{-k}$ is the mantissa.
    \end{itemize}
\end{definition}

Many programming languages use the IEEE 754 standard for floating point arithmetic. In this standard, for a double precision number, the base is $B=2$, the mantissa has $m=52$ 
bits and the exponent has $11$ bits.
\\
\\
Two additional numbers are added to the set of floating point numbers: $\pm \infty$ and NaN (Not a Number) which is used to represent undefined or unrepresentable values.
\begin{definition}
    The machine epsilon $eps$ is the smallest positive number which satisfies:
    \[
       | x -rd(x) | \leq eps | x |
    \]
    Where $rd(x)$ is the floating point representation of $x$. Usually this rounding function is taken to be 
    the nearest machine number.
\end{definition}

\setcounter{theorem}{2}
\begin{theorem}
    For a floating point number system with base $B$ and precision $m$, the machine epsilon is given by:
    The machine epsilon is given by $eps = B^{1-m}$ , i.e we have: 
    \[
        | x -rd(x) | \leq B^{1-m} | x |
    \]
\end{theorem}

\begin{theorem}
Let $\star$ be one of the operations $+, -, \times, /$ and let $\circledast$ be the equivalent floating point operation, then $\forall x, y$ in 
the floating point system, there exists an $\epsilon$ such that:
\[
    x \star y = (x \circledast y)(1 + \epsilon).
\]
\end{theorem}

\setcounter{definition}{11}
\begin{definition}
  Given the norms $\| \cdot \|_{(n)}$ and $\| \cdot \|_{(m)}$ on $\R^n$ and $\R^{m}$ respectively, we say that a matrix norm $\| \cdot \|_{\star}$ is \textbf{compatible} 
  with these norms if:
  \[
      \| A x \|_{(m)} \leq \| A \|_{\star} \| x \|_{(n)}, \quad \forall x \in \R^n.
  \]
\end{definition}
\bibliography{bib}

\end{document}

