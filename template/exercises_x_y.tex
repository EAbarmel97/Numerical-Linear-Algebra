\documentclass{exam}
\usepackage{babel}
\usepackage[utf8]{inputenc}
\usepackage{amsmath, amsthm, amssymb}
\usepackage{ragged2e}
\usepackage{lmodern}
\usepackage{tcolorbox}
\usepackage{hyperref}
\usepackage{verbatim}

\pagestyle{empty}
\renewcommand{\theenumi}{\alph{enumi}}
\renewenvironment{proof}{{\noindent\itshape\ignorespaces}}{{\hfill$\qed$\\}}

\DeclareMathOperator*{\argmin}{arg\,min}

\begin{document}

\begin{center}
    \textbf{\Large Problems x.y }
\end{center}

\section*{Exercise 1.1.1}
Provide a brief description of the problem for Exercise X.y.1.
\begin{enumerate}
    \item Prove some property.
    \item Discuss another aspect.
\end{enumerate}

\section*{Exercise x.y.2}
 Describe the problem for Exercise x.y.2.
\begin{enumerate}
    \item Prove a statement.
    \item Derive a formula.
\end{enumerate}

% % Solutions to other exercises in Chapter X...

\newpage

\begin{center}    
    \section*{SOLUTIONS}
\end{center}

\begin{comment}
    \subsection*{Exercise x.y.1 (a)}
    At vero eos et accusamus et iusto odio dignissimos ducimus qui blanditiis praesentium voluptatum deleniti atque corrupti quos 
    dolores et quas molestias excepturi sint occaecati cupiditate non provident, similique sunt in culpa qui officia deserunt mollitia animi, 
    id est laborum et dolorum fuga.
    \begin{proof}
        \begin{equation*}     
            \begin{aligned}
                \sigma^{\prime\prime\prime}(x) &= \frac{d}{ d x} \frac{e^{x}-e^{2x}}{(1 + e^{x})^3} = \frac{(e^x - 2e^{2x})(1 + e^x)^3 - 3(1 + e^x)^2 e^{x}(e^x - e^{2x})}{(1 + e^x)^6}\\
                &= \frac{e^{x} \{ 1 - 4e^x + e^{2x} \}(1 + e^x)^2}{(1 + e^x)^{6}} = \frac{e^{x} \{ 1 - 4e^x + e^{2x} \}}{(1 + e^x)^{4}}
            \end{aligned}
    \end{equation*}
    \end{proof}
    \subsection*{Exercise x.y.2 (a)}
    Et harum quidem rerum facilis est et expedita distinctio. Nam libero tempore, cum soluta nobis est 
    eligendi optio cumque nihil impedit quo minus id quod maxime placeat facere possimus, omnis voluptas assumenda est, 
    omnis dolor repellendus. 
    \begin{proof}
         a = a
    \end{proof}
\end{comment}
\end{document}